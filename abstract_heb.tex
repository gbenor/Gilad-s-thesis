\selectlanguage{hebrew}
\chapter*{תקציר}

מיקרו רנ"א הם רנ"א קצרים שאינם מקודדים, האחראיים על וויסות של ביטוי הגנים שלאחר התרגום וזאת ע"י התקשרות עם הרנ"א השליח. הפקה של צמדים של מיקרו רנ"א והרנ"א השליח אינה פשוטה ומלווה באתגרים טכניים מורכבים. בנקודת הזמן הנוכחית, קיימים מאגרי נתונים של צמדי מיקרו רנ"א ורנ"א שליח רק עבור מספר קטן של אורגניזמים. בשנים האחרונות, פותחו שיטות חיזוי מבוססות למידת מכונה לחיזוי התקשרויות של מיקרו רנ"א. היכולת להשתמש בשיטות והמודלים הללו ולהחיל אותם על אורגניזמים שעבורם אין נתונים תהווה תוספת משמעותית למחקר. עם זאת, לא ידוע עדיין כיצד להחיל את השיטות הללו, ואילו מהמאפיינים של ההתקשרויות משותף לכלל האורגניזמים ולמעשה נשמרים באבולוציה.

\vspace{5mm} %5mm vertical space

בעבודה זו, השתמשנו בשמונה מאגרי נתונים של אינטראקציות ישירות מארבעה אורגניזמים: בני אדם, עכבר, תולעת ובקר כדי לבדוק האם כללי ההתקשרות ניתנים להעברה מאורגניזם אחד למשנהו. בעזרת מסווגים מבוססי למידת מכונה, השגנו דיוק גבוה לסיווג נתונים בתוך אותו מין, וגילינו שהתכונות המשפיעות ביותר חופפות משמעותית בין האורגניזמים השונים. ביצענו מחקר מקיף של מערכות היחסים בין מאגרי הנתונים השונים שבידינו לרבות מדידת פילוג ופיזור של משפחות מיקרו רנ"א והערכת הביצועים של סיווג בין כל הזוגות האפשריים של מאגרי הנתונים שבידינו. הראנו בעבודה שישנה קורלציה בין הביצועים של המסווגים לבין המרחק האבולוציוני בין האורגניזמים ולבין המרחק בין המשפחות השונות.

\vspace{5mm} %5mm vertical space


תוצאות אלה מצביעות על כך שההעברה של כללי יצירת אינטראקציות בין אורגניזמים תלויה במספר גורמים, כולל הרכב משפחות מיקרו-רנ"א והמרחק האבולוציוני. יתר על כן, אנו מציגים כי חלק מהתכונות משותפות וחופפות בין אורגניזמים שונים. המחקר מספק תובנות חשובות פיתוחים עתידיות של כלי חיזוי גני מטרה שניתן ליישם על אורגניזמים ללא נתונים ניסויים מספקים.


\selectlanguage{english}
