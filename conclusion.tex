\chapter{Discussion and Conclusion}
\label{chap:conclusion}
Identification of bona-fide miRNA targets is crucial for elucidating the functional roles of miRNAs and remains a major challenge in the field. Notable progress in this task has been achieved due to novel experimental protocols that can produce high-throughput unambiguous interacting miRNA-target datasets. However, due to technical challenges involved in the application of these methods, there is a constantly increasing interest in using computational approaches for miRNA target prediction, especially those that are based on advanced machine learning models. Several studies successfully trained and applied classic machine learning \cite{lu2016learning, ding2016tarpmir, wang2016improving, liu2019prediction} and deep learning \cite{wen2018deepmirtar, paker2019mirlstm, pla2018miraw} methods on some of the experimental miRNA-target datasets from a few model organisms. However, our limited understating of the evolution of miRNA-target interactions, puts in question the applicability of these tools to organisms with no available experimental training data.

The ultimate goal of this study was to evaluate the transferability of miRNA-target rules between the examined organisms, as well as to identify and compare their most influential interaction features. To this end, we systematically characterized the available miRNA-target chimeric datasets and conducted intra- and cross- datasets classification analyses using machine learning approaches.  

\section{Available data}
The availability of large and high-quality data is crucial for machine learning-based research. In the field of experimental miRNA-target identification, several approaches to generate high-throughput datasets are available, each one has its benefits and limitations \cite{li2019current, martinez2013microrna}. 

In our analysis, we focused on chimeric miRNA-target datasets, generated by experimental or endogenous ligation (by e.g., CLASH \cite{helwak2013mapping} or PAR-CLIP \cite{grosswendt2014unambiguous}), as these datasets provide direct evidence for interactions between a miRNA and a specific target site. Furthermore, these datasets contain many non-canonical interactions that enrich the repertoire of miRNA-target interactions. On the other hand, the main limitation of ligation-based methods is the low yield of chimeric reads that are being recovered (around $\sim 2$\%), suggesting that a large number of miRNA-target interactions remains uncaptured. In this work, we assume that the captured interactions represent an unbiased sampling of all the interactions in the examined cells. Additional advances in the efficiency of ligation-based methods and deeper sequencing will provide richer datasets that would be easily incorporated into our analysis for further research.  

We utilized eight available chimeric datasets from four organisms (worm to human), that were generated by different experimental protocols (Table \ref{tbl:dataset_description}). We developed a processing pipeline to transform and unify the different data formats that we encountered during the collection of the datasets. This pipeline is a powerful infrastructure that will enable us, with a relatively low effort, to add more data sources to the analysis in the future, when these become available. 

\section{Thorough analysis of the datasets}
We characterized the datasets based on their miRNA content and base-pairing patterns.
Our analysis of the frequencies of miRNA sequences revealed that there are differences in miRNA sequence distributions among datasets, even if they originated from the same organism. In addition, each dataset is dominated by a small set of miRNAs (30-50\% of most frequent miRNAs comprise 90\% of the interactions). These distributions mirror the in vivo distributions, as it was reported that miRNA frequency in miRNA-target chimeras is correlated with total miRNA abundance \cite{darnell_moore2015mirna}.

We continued categorizing the interactions based on their seed-pairing type (canonical and non-canonical) and base-pairing density. Perfect seed complementarity (referred to as canonical seed pairing) between target sites and miRNA seed sequences (nts 2-7 or 2-8), has long been recognized as a critical dominant feature which determines miRNA targeting efficiency \cite{bartel2009micrornas, lewis2005conserved, schirle2014structural}. Nevertheless, in recent years several examples of functional miRNA-target interactions without perfect seed pairing have been reported, featuring \textit{GU} pairs, mismatches, and bulges in the seed region (referred to as non-canonical seed pairing). Examples include the well-established \textit{let-7} targeting of \textit{lin-41} in \textit{C. elegans} \cite{slack2000lin, vella2004c}, with one site containing one-nucleotide bulge in the target, and the other site containing a \textit{GU} pair. Moreover, non-canonical miRNA-target sites known as “nucleation bulges”, in which the target sites contain a bulged-out \textit{G} in the seed, were identified for \textit{miR-124} when analyzing AGO HITS-CLIP data from mouse brain \cite{chi2012alternative}. The functionality of non-canonical sites is still a matter of debate. While studies that generated miRNA-target chimeras provided evidence for the functionality of the recovered non-canonical interactions \cite{helwak2013mapping,grosswendt2014unambiguous}, a recent analysis of non-canonical target sites revealed that even though these sites are bound by the miRNA complex, they do not appear to be broadly involved in the regulation of gene expression \cite{agarwal2015predicting}. Future work will need to focus on generating miRNA functional high-throughput datasets \cite{soriano2019functional} across organisms, that would be combined with datasets of chimeric interactions, to provide a more robust starting point for similar types of studies.


We showed that the majority of the interactions in most datasets are non-canonical (48-70\%). Furthermore, in both canonical and non-canonical groups, a large fraction of the interactions is characterized by a medium and a high density of base-pairing (11-16 and more than 16 base-pairs, respectively), predicting the existence of additional pairing beyond the seed region. These auxiliary non-seed interactions were suggested to compensate for imperfect seed matches \cite{brennecke2005principles, grimson2007microrna}. Moreover, non-seed interactions were also shown to contribute to target specificity among miRNA seed family members (same seed, divergent non-seed sequence), both in the case of canonical and non-canonical seed pairings \cite{broughton2016pairing, darnell_moore2015mirna}.


\section{Features and their significance}
In this work, we partially adapted the pipeline from DeepMirTar \cite{wen2018deepmirtar}. In DeepMirTar the interactions are represented by 750 features. These features include high-level and low-level expert-designed features that represent the interacting duplex, sequence composition, free energy, site accessibility and conservation. Additional raw-data-level features encode the sequences of the miRNA and the target site. We have adopted some of the expert-designed features in our study and used a total of 490 various features to describe the interactions, enabling the model to identify and learn different interaction patterns. 

We did not include, however, the previously suggested raw-data-level features, to avoid potential information-leakage from the training set to the testing set. First, we saw that the miRNA seed families are not uniformly distributed. Second, in our study, the negative sequences are synthetically generated such that there is no match in the seed region to any annotated miRNA. Thus, including these features could lead the machine learning model to learn to distinguish between real and mock miRNA seeds.  Moreover, in such a case, the model may be over-fitted and fail to generalize the rules of interactions. Indeed, and perhaps not surprisingly, we achieved higher classification performance when we included the raw-data-level features in our models (Table \ref{tab:resultswithrawfeature}). Another study \cite{pla2018miraw} that used raw sequence features addressed this issue by generating a negative dataset based on experimentally verified data instead of using mock miRNAs. A comparison between different methods for the generation of negative datasets is an interesting direction for future research. In particular, the evaluation of how the combination of these methods and different feature sets affects the performance of miRNA-target prediction classifiers would help to generate standard approaches for future studies.

The feature importance analysis revealed that there is a small group of significantly dominant features in all datasets. Even though the analysis identified the features for each dataset independently, we showed that there is a significant overlap between the groups, and the unified group contains only 16 features (Table \ref{tab:feature_importance}). Importantly, half of these features are seed-related, reiterating the significance of this region in miRNA-target interactions \cite{agarwal2015predicting}.

Ideally, in machine learning, we want the ratio between samples and features to be high enough to have a robust model and avoid over-fitting. Some of the datasets in our collection are relatively small, with a low ratio of interactions to features. For \textit{ce1, ce2, h2} the ratio is $\sim$4, and for \textit{m1} it is $\sim$2. Low ratio can produce models with high bias and high variance. In general, a reduction in the number of features, when possible, was shown to be a successful practice. In this work, some of the features are highly correlated, thus can be combined. There are several methods for feature selection and dimensionality reduction that may be evaluated in the future. As a preview, we used a basic method for feature selection, based on the XGBoost feature importance data. We used the 16 features taken from Table \ref{tab:feature_importance} and repeated the classification analysis (Table \ref{tab:self_summary16}, Figure \ref{fig:crossdataset16}). The results were similar to the results obtained when all features were included, indicating that future research that will evaluate different dimensionality reduction methods should be considered to optimize the classification models. 


\begin{table}[h!]
\caption{Intra-dataset classification accuracy of xGBoost classifier trained on an extended 580 features set (including one-hot encoding features)}
\label{tab:resultswithrawfeature}
\centering
\begin{tabular}{|l|l|}
\hline
\textbf{Dataset} & \textbf{Accuracy}      \\ \hline
ca1     & 0.985 (0.001) \\ \hline
ce1     & 0.951 (0.006) \\ \hline
ce2     & 0.959 (0.008) \\ \hline
h1      & 0.93 (0.006)  \\ \hline
h2      & 0.922 (0.008) \\ \hline
h2      & 0.908 (0.007) \\ \hline
m1      & 0.877 (0.014) \\ \hline
m2      & 0.956 (0.002) \\ \hline
\end{tabular}
\bigbreak
\caption*{The cells contain the mean and the standard deviation (in brackets) values acquired from 20 models that were trained and evaluated on different training-testing stratified dataset splits.}
\end{table}



\section{Training and testing dataset split} 
The splitting procedure of the data into training and testing sets has a crucial role in the evaluation of machine learning models. In the miRNA-target prediction task, there is no pre-defined split to training and testing sets as is usually common in other fields, for example, in computer vision (e.g., MNIST \cite{mnist10027939599}). Therefore, we used three strategies to reduce the effect of the split on our results: (1) using stratified training-testing split which ensures the same distribution of miRNA sequences in both training and testing sets; (2) generating control sets by a pure-random split algorithm and (3) generating for the above split approaches several training-testing sets using different random states and reporting the mean and the standard deviation values of the results. Indeed, we got similar results with both splitting methods and very low standard deviation values, reassuring that the split strategies did not bias our results.

\section{Using tree-based classifier} 
For our thorough analysis, we used XGboost \cite{xgboost}, which is one of the leading gradient boosting tree-based tools for classification \cite{nielsen2016tree}. Differently from deep-learning, XGboost is less computationally expensive and usually does not require a GPU for training, and it can work both with small and large datasets. Additionally, XGboost provides the ability to evaluate and explain the classification rules and rank the features by their importance. 
We showed that XGboost achieved the best performance over the statistical machine learning algorithms (e.g., SVM and LR) for all datasets. Furthermore, it achieved comparable results to deep learning algorithms that were previously applied on the human dataset \textit{h1} \cite{wen2018deepmirtar, lee2016deeptarget}.

\section{Cross-dataset analysis}
Most of the previous works trained and tested their predictive models based on a single chimeric miRNA-target dataset (usually \textit{h1}), sometimes complemented by additional experimental data from databases (e.g., \cite{xiao2009mirecords,chou2016mirtarbase}) or AGO-CLIP data \cite{ding2016tarpmir,wen2018deepmirtar,paker2019mirlstm, lu2016learning, pla2018miraw}. Then these models were evaluated on portions taken out from the training set and in some cases on a few independent datasets from the same or other organisms  (Table \ref{tab:toolsummary}). 
The contribution of our work is in providing for the first-time a thorough analysis of all available miRNA-target chimeric datasets, outlining their similarities and dissimilarities. Additionally, we explored the ability to learn classification rules from one dataset and apply them on another dataset, considering all possible combinations of dataset pairs.

The accuracy results of cross-dataset classification ranged between 0.56 to 0.94. To be able to explain these results we examined several factors:

(1) Evolutionary distance - we estimated the distance for every pair of organisms (i.e., the time since the organisms diverged from their common ancestor (Table \ref{tab:evolutiontime})). Among the organisms, the mouse and the human are the closest, with cattle equally and relatively close to them, while \textit{C. elegans} is the most distant from all. Indeed, we got the highest accuracy results when we trained-tested datasets of the same organism and the lowest accuracy results when we trained-tested combinations of \textit{C.elegans} datasets with those from other organisms.

(2) Kullback–Leibler (KL) divergence scores -  we measured the divergence for every pair of datasets based on their miRNA seed family distribution. Previous analysis of chimeric datasets showed that individual miRNAs exhibit enrichment for specific classes of base-pairing patterns \cite{helwak2013mapping, broughton2016pairing}, suggesting that they may follow different targeting rules.
Thus, differences in the distributions of miRNA sequences in the training sets may lead to biases in the rules learned by a machine learning model, which could partially explain the high correlation that we observe between KL-divergence and the classification performance.  

Interestingly, and maybe not surprisingly, the KL divergence results coincide with the evolutionary distance of the organisms, where \textit{C.elegans} datasets exhibit the highest distance from the datasets of other organisms. The divergence within the same organism is, on average, lower than the divergence between different organisms. This divergence is probably associates with the differences in miRNA distributions among different cell types or developmental stages from which the datasets were generated (Table \ref{tbl:dataset_description}). 

(3) Area covered in 2D feature space - We visualized the datasets by their features in 2 dimensions using PCA. The visualization highlighted datasets with a lower spread. In particular, \textit{C. elegans} datasets are exceptional relative to the rest of the organisms, concentrated in a narrower area. In addition, the datasets \textit{m1} and \textit{h2}, which represent endogenously ligated chimeras from a mixture of AGO-CLIP experiments, have relatively smaller sizes compared to other datasets from the same organism and have a lower spread. The exception of these two datasets may explain the lower accuracy results obtained in cross-datasets experiments when we used them as training sets.


\begin{table}[h!]
\caption{Estimated divergence time {[}MYA{]} between organisms in our study}
\label{tab:evolutiontime}
\begin{tabular}{|l|l|l|l|}
\hline
             & Mouse & Cattle & C.elegans \\ \hline
Human & 90  & 96         & 797                    \\ \hline
Mouse          &     & 96         & 797                    \\ \hline
Cattle   &     &            & 797                    \\ \hline
\end{tabular}
\caption*{Each cell represents the time since the pair of organisms from the corresponding row and column diverged from their common ancestor (source: \cite{kumar2017timetree})}
\end{table}



\section{Conclusions and future perspectives}
The accuracy results obtained in our cross-datasets experiments are pretty high, as long as the organisms are within a certain evolutionary distance, reflecting the ability of the machine learning model to generalize interaction rules learned from a specific datastet, into more universal interaction rules. Altogether our results suggest that target prediction models could be applied also to organisms where experimental training data is limited or unavailable, as long as they are close enough to the organism that is used for training.
As more miRNA-mRNA interaction datasets become available, they could be processed with our pipeline and incorporated into the cross-dataset analysis. Expansion of such analysis on more datasets in the future may also provide insights about the evolution of miRNA-targeting, and identification of general, as well as organism-specific features. Another interesting research direction will be to combine the information from several datasets in an iterative manner and examine the prediction accuracy in close and more distant organisms.
