\chapter{Introduction}
\label{chap:intro}

% \subsection*{miRNAs summary}
MicroRNAs (miRNAs) are small non-coding RNAs that regulate gene expression post-transcriptionally. miRNAs are encoded in the genome and are generated in a multi-stage process by endogenous protein factors \cite{finnegan2013microrna}. The functional mature miRNAs associate with Argonaute proteins to form the core of the miRNA-induced silencing complex (miRISC). miRISC uses the sequence information in the miRNA as a guide to recognize and bind partially complementary sequences on 3' untranslated region (UTR) of target mRNAs. miRISC binding typically leads to translational inhibition and/or degradation of targeted mRNAs \cite{huntzinger2011gene}. 
miRNAs are conserved throughout evolution and are present in the genomes of animals and plants \cite{kozomara2013mirbase}. miRNAs have diverse functions in development and physiology and have been implicated in many human diseases \cite{rupaimoole2017microrna}.

%\subsection*{Identification of miRNA-target interactions}
The identification of miRNA target sites on mRNAs is a fundamental step in understanding miRNA involvement in cellular processes. Several experimental high-throughput methods for identifying miRNA targets have been developed in recent years \cite{li2019current, martinez2013microrna}.
The most common and straightforward approach is based on measuring changes in mRNA levels following miRNA over-expression or inhibition in tissue-cultured cells \cite{thomas2010desperately}. However, this approach has several major limitations \cite{li2019current, martinez2013microrna}. First, such data may contain indirect signals of miRNA regulation from the downstream genes of direct miRNA targets. Second, for direct regulation, the exact sequences of binding sites are unknown and have to be predicted within the relevant mRNA sequence. Furthermore, such experimental settings may represent a non-physiological context for miRNA activity which does not reflect endogenous targeting rules. Finally, it may miss signals of translation efficiency inhibitions which consequently affect gene expression but are not reflected in changes in mRNA levels \cite{fabian2010regulation}.


Other methods, e.g., HITS-CLIP \cite{chi2009argonaute, zisoulis2010comprehensive} and PAR-CLIP \cite{hafner2010transcriptome}, are based on crosslinking and immunoprecipitation (CLIP) of RNA-protein complexes that are found in direct contact. The crosslinked complexes are immunoprecipitated with a specific AGO antibody (AGO-CLIP), and the associated miRNAs and mRNA targets are collected for further sequencing analysis. Though these methods greatly decrease the target search space to precise regions on mRNAs, the identity of the specific miRNA engaged in each interaction is unknown and has to be predicted bioinformatically \cite{wang2015design, uhl2017computational}, by e.g., identifying which highly expressed miRNAs are associated with individual AGO-CLIP peaks \cite{majoros2013microrna, reczko2012functional, liu2013clip, khorshid2013biophysical}.


Recently, more advanced methods, e.g., CLASH (Cross-linking, Ligation and Sequencing of Hybrids) \cite{helwak2013mapping}, CLEAR (covalent ligation of endogenous Argonaute-bound RNAs)-CLIP \cite{darnell_moore2015mirna, scheel2017global} and modified iPAR-CLIP \cite{grosswendt2014unambiguous}, have been developed to capture miRNAs bound to their respective targets. These methods are derived from AGO-CLIP and use an extra step to covalently ligate the 3' end of a miRNA and the 5' end of the associated target RNA within the miRISC. Subsequent cloning and sequencing of isolated chimeric miRNA-target reads facilitate the identification of direct miRNA-target interactions. Using these methods, datasets of chimeric miRNA-target interactions were generated from cells originating from human, mouse, the nematode \textit{Caenorhabditis elegans (C. elegans)}, and the cattle \textit{Bos taurus}.
An additional method, iCLIP \cite{broughton2016pairing}, was applied to \textit{C. elegans} to recover chimeric sequences without employing the ligation step. Furthermore, re-analysis of published human and mouse AGO-CLIP data discovered additional miRNA-target chimeric interactions in libraries where no ligase was added \cite{grosswendt2014unambiguous}. 

The analysis of chimeric miRNA-target interactions from the above studies revealed that many of them display non-canonical seed binding patterns and involve nucleotides outside of the seed region. Despite the great contribution that these experimental methods can bring to the miRNA field, their application is technically challenging. Thus so far, datasets were generated for only a small number of model organisms (Table \ref{tbl:dataset_description}).

%\subsubsection*{Computational miRNA-target prediction}
The limited number of experimentally identified miRNA-target interactions promoted the use of computational predictions to expand miRNA-target repertoires. Nevertheless, computational identification is very challenging, since miRNAs are short and engage only a partial sequence complementarity to their targets, and the rules that govern the miRNA targeting process are not yet fully understood. 
Over the past 15 years, many computational tools were developed for miRNA target prediction. The first generation of tools was based on very general rules of thumb, e.g., canonical seed pairing, miRNA-target duplex energy, conservation of the target site and accessibility \cite{kruger2006rnahybrid, enright2003microrna, lewis2005conserved, kertesz2007role}. These tools suffer from high False Positive and False Negative prediction rates \cite{pinzon2017microrna, oliveira2017combining, fridrich2019too, min2010got}, due to limitations of general rules and insufficient knowledge about seedless interactions and base pairing patterns in the non-seed region. In addition, the target prediction outputs of various tools only partially overlap, making it difficult to choose candidates for further experimental validation or more global downstream analysis.


The availability of new datasets of high-throughput, direct miRNA-target interactions (e.g., \cite{scheel2017global, grosswendt2014unambiguous, darnell_moore2015mirna, helwak2013mapping}), led to the development of new machine-learning (ML) based methods for miRNA target prediction \cite{lu2016learning, ding2016tarpmir, pla2018miraw, wen2018deepmirtar, cheng2015mirtdl, menor2014mirmark}. These ML-based methods were designed to capture both canonical sites based on seed complementarity and non-canonical sites with pairing beyond the seed region. These methods incorporated in their models tens to hundreds of different features to represent e.g., sequence, structure, conservation, and context of the interacting molecules and were reported to achieve significant improvement in overall predictive performance than the previous tools. Differences in several aspects can be observed among ML-based methods, including the ML approach and the features used, the choice of datasets for training and testing, inclusion or exclusion of non-canonical interactions from the training/testing set, and the approach of generating negative data. 



The experimental datasets that are used to train the ML-based tools are limited to only a few model organisms. Nevertheless, there is a need to apply target prediction tools to other organisms as well, where experimental data is not available. Though some of the ML methods examined the possibility to predict interactions in organisms different from the organism they were trained on, in all cases the training was performed on human datasets and was applied on a few other organisms. The ability to do the predictions in the opposite direction, or between organisms other than human was not tested. Moreover, it is largely unknown how the patterns of miRNA-target interactions evolve across bilaterian species and whether there are features that have been fixed during evolution, thus questioning the general applicability of these ML methods across species.  

In this work, we used available datasets of high-throughput direct miRNA-target interactions to explore whether miRNA-target interaction rules are transferable from one organism to another. We evaluated eight datasets from four organisms (human, mouse, worm, and cattle), generated from various tissues and experimental protocols. We developed a processing pipeline to transform these datasets into a standard format that enables their comparison and integration. We provide a detailed overview of the datasets, focusing on their sizes, miRNA-seed families composition, and interaction patterns while highlighting their resemblance and dissimilarity. For each dataset, we trained and tested 6 commonly used machine learning classifiers for the prediction of miRNA-target interactions and evaluated the importance of the features we used.
We then explored the relationships between datasets by measuring the divergence of their miRNA seed sequences, and by evaluating the performance of cross-datasets classification. Our results indicate that the transferability of miRNA-targeting rules between different organisms depends on several factors, including the composition of seed families and evolutionary distance. Our study provides important insights for future developments of target prediction tools that could be applied to organisms without sufficient experimental data.