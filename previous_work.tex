\chapter{Previous Work}
\label{chap:pw}

Several ML-based methods utilized chimeric miRNA-target datasets to build and evaluate their models. These methods differ in several aspects, including the ML approach and the features used, the choice of datasets for training and testing, inclusion or exclusion of non-canonical interactions from the training/testing sets, and the generation of negative data. In the following sections and in Tables \ref{tab:toolsummary}, \ref{tab:existingmethods}, we provide a summary of some of these methods, focusing on these aspects.

\section{Machine learning method and features}
The method chimiRic \cite{lu2016learning}, combines two support vector machine (SVM) classifiers, one to predict miRNA-mRNA duplexes and a second to learn AGO binding preferences. Features used by this method are designed to describe the duplex structure (e.g., binding information at each miRNA position \cite{khorshid2013biophysical}, positions of loops' opening and extension), UTR local sequence context, and global positional context. 

TarPmiR \cite{ding2016tarpmir} is a random-forest-based approach that provides a probability that a candidate target site is a true target site. It integrates six conventional features (e.g., folding energy and seed-matching) and seven new features (e.g., largest consecutive pairings).

DeepMirTar \cite{wen2018deepmirtar} is based on a stacked de-noising auto-encoder deep learning method (SdA) and uses 750 different features to describe the interactions (including raw-data-level and expert-designed high-level and low-level features). These features capture the information about the seed match, sequence composition, free energy, site accessibility, conservation, and hot-encoding of miRNAs and their target sites. 

miRAW \cite{pla2018miraw} relies on deep artificial neural networks (ANN) to automatically learn the relevant features describing miRNA-gene interactions for predicting miRNA targets. The method works with raw input data and makes no assumptions about suitable input descriptors. 

mirLSTM \cite{paker2019mirlstm} is a deep learning approach that is based on Long Short Term Memory (LSTM). The method captures only the information encoded in the duplex formed between the miRNA and its relevant binding site, which is reduced into a vector over 5 letter alphabet to express four possible base-pairs (\textit{AU,UA,GC,CG}) and one letter for all remaining combinations (including \textit{GU}, bulge, mismatch). These vectors are converted into embedded words to find the relationship between similar sequences using Euclidean distance. Then, the embedded words are fed to the first layer of their LSTM architecture. 

MirTarget \cite{wang2016improving} is based on SVM and uses 50 features that include nucleotide composition, accessibility of the target site, seed conservation, seed base-pairing stability, and location of the target site. In a subsequent study \cite{liu2019prediction}, the model was refined with 96 features. 

\section{Training and testing datasets}
All the above studies trained and tested their models on a dataset of chimeric interactions from human cells generated with the CLASH method \cite{helwak2013mapping}, referred to as \textit{h1} in the manuscript. In some of the studies, the dataset was filtered based on the location of the sites, seed pairing pattern, or functional evidence. In other studies, it was complemented with additional interactions from other experiments.

DeepMirTar \cite{wen2018deepmirtar} and mirLSTM \cite{paker2019mirlstm} filtered this dataset to include only canonical and non-canonical sites that are located at the 3'UTRs. They complemented this dataset with an additional small number of interactions retrieved from miRecords \cite{xiao2009mirecords}). TarPmir \cite{ding2016tarpmir} used all the interactions available from human CLASH dataset \textit{h1}.
chimiRic \cite{lu2016learning} was trained on a combination of human CLASH dataset and seed-containing sites from AGO-CLIP data. The model was tested on specific miRNA families, by holding out their interactions from the training set. For miRAW \cite{pla2018miraw} the training set was built by intersecting human CLASH, AGO-CLIP and TargetScanHuman \cite{agarwal2015predicting} datasets with mirTarBase \cite{chou2016mirtarbase} and Diana TarBase \cite{vlachos2015diana} to include only validated functional interactions. The initial evaluation was performed on experimentally verified miRNA:gene-targets that were excluded from the training set. 
miRTarget  \cite{wang2016improving} was trained and tested on human CLASH data combined with chimeras generated by endogenous ligation in human AGO-CLIP experiments \cite{grosswendt2014unambiguous}. miRTarget v4 \cite{liu2019prediction} combined human chimeras with miRNA overexpression data to identify common features that are characteristic of both miRNA binding and target downregulation.

%%%%%%%%%%%%from supppp

\section{Independent testing datasets}
For additional independent testing, the above methods used various datasets, not necessarily derived from ligation-based experiments. DeepMirTar \cite{wen2018deepmirtar} was tested on human PAR-CLIP dataset. mirLSTM \cite{paker2019mirlstm} was tested on a small experimental dataset. TarPmir \cite{ding2016tarpmir} was tested on two PAR-CLIP human datasets and a HITS-CLIP dataset from mouse. chimiRic's \cite{lu2016learning} model was tested on chimeric interactions from  \textit{C. elegans} \cite{grosswendt2014unambiguous} and mouse \cite{darnell_moore2015mirna}. 
miRAW \cite{pla2018miraw} performed additional evaluations on microarray datasets reporting mRNA changes after transfecting a miRNA into HeLa cells. miRTarget \cite{wang2016improving} and miRTarget v4 \cite{liu2019prediction} tested their model on mRNA microarray datasets from miRNA knockdown experiments. miRTarget v4 performed additional testing on HITS-CLIP data from the mouse brain.

\section{Negative datasets}
For negative datasets, DeepMirTar \cite{wen2018deepmirtar} and mirLSTM \cite{paker2019mirlstm} used mock miRNAs to generate a negative interaction for each positive one. TarPmir \cite{ding2016tarpmir} generated negative target sites on positive mRNAs, such that they do not overlap with positive sites and have similar nucleotide composition compared to the positive sites. miRAW built a negative dataset based upon experimentally verified data. chimiRic \cite{lu2016learning} generated negative set from canonical miRNA seed matches that are not AGO bound based on CLIP data, together with (miRNA, site) pairs where an AGO-bound site is paired with an incorrect miRNA based on CLASH chimeras. miRTarget \cite{wang2016improving} and miRTarget v4 \cite{liu2019prediction} generated negative examples based on CLIP data, selecting non-target sites based on a set of criteria, e.g., no overlap with a positive site, detectable expression of the transcript based on microarrays, and the existence of perfect seed match to one of the miRNAs expressed in the cells.



\clearpage
\begin{landscape}
\begin{table}[h!]
\centering

\caption{A summary of machine-learning based methods that utilized chimeric miRNA-target datasets in their models.}
\label{tab:toolsummary}
\centering
   \resizebox{\textwidth}{!}{%

\begin{tabular}{|l|l|l|l|l|l|}
\hline
\textbf{Tool} & \textbf{ML method} & \textbf{Datasets used for training/testing}                                     & \textbf{Independent Dataset}                                                   & \textbf{Negative interactions} & \textbf{Features} \\ \hline
chimiRic \cite{lu2016learning}      & SVM                & Human (CLASH, AGO-CLIP)                                                                                            & \begin{tabular}[c]{@{}l@{}} chimeras from \\ Mouse and \textit{C.elegans} \end{tabular}                                                              & \begin{tabular}[c]{@{}l@{}}seed matching \\ non-CLIP sites \end{tabular}                   & Small             \\ \hline
TarPmiR \cite{ding2016tarpmir}       & Random Forest      & Human (CLASH)                                                                                                         & \begin{tabular}[c]{@{}l@{}}Human (PAR-CLIP), \\ Mouse (HITS-CLIP)\end{tabular}   & \begin{tabular}[c]{@{}l@{}}Negative target sites, \\ on positive mRNAs\end{tabular}                     & 13                \\ \hline
DeepMirTar \cite{wen2018deepmirtar}   & Deep learning      & Human (CLASH) + mirRecords                                                                                            & Human (PAR-CLIP)                                                                & Mock miRNAs                          & 750               \\ \hline
miRAW \cite{pla2018miraw}        & Deep learning      & \begin{tabular}[c]{@{}l@{}}Human (intersection of: CLASH, CLIP, \\ TargetScan with Diana TarBase, mirTarBase)\end{tabular} & Human (microarray)                                                              & Experimental data                  & Raw sequence       \\ \hline
mirLSTM \cite{paker2019mirlstm}      & Deep learning      & Human (CLASH) + mirRecords                                                                                            & Experimental                                                                   & Mock miRNAs                          & Raw sequence       \\ \hline
mirTarget \cite{wang2016improving}    & SVM                & Human (CLASH, AGO-CLIP)                                                                                               & Human(microarrays)                                                              & \begin{tabular}[c]{@{}l@{}}seed matching non-CLIP sites \\ on expressed mRNAs \end{tabular}                           & 50                \\ \hline
mirTarget v4 \cite{liu2019prediction} & SVM                & \begin{tabular}[c]{@{}l@{}}Human (Intersection of \\ CLASH and microarrays) \end{tabular}                                                                                             & \begin{tabular}[c]{@{}l@{}}Mouse (HITS-CLIP), \\ Human (microarrays)\end{tabular} & \begin{tabular}[c]{@{}l@{}}seed matching non-CLIP sites \\ on expressed mRNAs \end{tabular}                          & 96                \\ \hline
\end{tabular}}
\end{table}
\end{landscape}

\bigbreak
\bigbreak


\begin{table}[h!]
\caption{The accuracy performance of different tools (left) and machine learning methods (right), on human dataset referred to as \textit{h1} in the manuscript, taken from deepMirTar \cite{wen2018deepmirtar}.
}
\label{tab:existingmethods}
\centering
   \resizebox{\textwidth}{!}{%

\begin{tabular}{lll|l|l|}
\cline{1-2} \cline{4-5}
\multicolumn{1}{|l|}{\textbf{Methods}} & \multicolumn{1}{l|}{\textbf{ACC}} & \textbf{} & \textbf{Machine learning alg.} & \textbf{ACC}    \\ \cline{1-2} \cline{4-5} 
\multicolumn{1}{|l|}{Miranda}          & \multicolumn{1}{l|}{0.6592}       &           & DT (Decision Tree)                            & 0.8139 (0.0137) \\ \cline{1-2} \cline{4-5} 
\multicolumn{1}{|l|}{RNAhybrid}        & \multicolumn{1}{l|}{0.6988}       &           & BNB (Bernoulli Naïve Bayes)                            & 0.7570 (0.0098) \\ \cline{1-2} \cline{4-5} 
\multicolumn{1}{|l|}{PITA}             & \multicolumn{1}{l|}{0.4981}       &           & LR (Logistic Regression)                            & 0.8491 (0.0117) \\ \cline{1-2} \cline{4-5} 
\multicolumn{1}{|l|}{TargetScan v7.0a} & \multicolumn{1}{l|}{0.5801}       &           & RF (Random Forest)                             & 0.8811 (0.0090) \\ \cline{1-2} \cline{4-5} 
\multicolumn{1}{|l|}{TarPmiR}          & \multicolumn{1}{l|}{0.7446}       &           & MLP (Multi-Layer Perceptron)                            & 0.8990 (0.0099) \\ \cline{1-2} \cline{4-5} 
\multicolumn{1}{|l|}{DeepMirTar (SdA)} & \multicolumn{1}{l|}{0.9348}       &           & CNN-1D (Convolutional Neural Network)                         & 0.8886 (0.0145) \\ \cline{1-2} \cline{4-5} 
                                       &                                   &           & CNN-2D (Convolutional Neural Network)                         & 0.8765 (0.0169) \\ \cline{4-5} 
\end{tabular}}
\bigbreak
\caption*{The right table contains the mean and the standard deviation values (in brackets) acquired from 20 models that were built based on different random splits of the dataset to training/validation/testing.}
\end{table}